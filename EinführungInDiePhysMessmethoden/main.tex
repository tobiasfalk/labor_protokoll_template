\documentclass[12pt,a4paper,twoside]{article}

\usepackage{labor}


\begin{document}



%fill for cover and header creation
\title{Oszillograph}
\newcommand\supervisor{Mustermann, Max}
\newcommand\groupnumber{9}

\newcommand\participantonelastname{Falk}
\newcommand\participantonefirstname{Tobias}
\newcommand\participantoneid{12303625}
\author{\participantonefirstname \ \participantonelastname}

\newcommand\semester{SS 2024}
\date{2024-03-05}

\pagenumbering{Roman}

\begin{figure*}
	\includepdf{Deckblatt_EPM}
	\begin{tikzpicture}
		\draw(0,0);
		\draw(210mm,-297mm);
		\node[anchor=west] at (25mm,-155mm) {\Large \textbf{\thetitle}};
		\node[anchor=west] at (25mm,-166.5mm) {\Large \textbf{\supervisor}};
		\node[anchor=center, scale = 5] at (57mm,-192.5mm) {\textbf{\groupnumber}};
		\node[anchor=west] at (25mm,-166.5mm) {\Large \textbf{\supervisor}};
		\node[anchor=west] at (20mm,-210mm) {\Large \textbf{\participantonefirstname \ \participantonelastname}};
		\node[anchor=west] at (38mm,-221.5mm) {\Large \textbf{\participantoneid}};
		\node[anchor=west] at (25mm,-239mm) {\Large \textbf{\thedate}};
		\draw[fill = white, white] (112mm, -235mm) rectangle (160mm, -242.3mm); % 
		\node[anchor=west] at (120mm,-239mm) {\Large \textbf{\semester}};
	\end{tikzpicture}
\end{figure*}

%\newpage\null\thispagestyle{empty}\newpage

\clearpage

\newpage\null\thispagestyle{empty}\newpage

\pagenumbering{arabic}
\setcounter{page}{1}




%\maketitle %short title alternative



\tableofcontents
\newpage



%\section*{Anmerkung}

%Dies ist ein 3rd-Hand Protokoll. Das Experiment wurde nicht vom Autor durchgeführt, sondern hat dieser eine Videoaufzeichnung \cite{teachcenter} vom Experiment protokolliert. Weiters stammen das Titelblatt (vom Autor ausgefüllt) und die ersten drei Kapitel (inhaltlich) ebenfalls aus diesem Verzeichnis.

%Aufgrund der herrschenden COVID-19-Maßnahmen, wurde die Laborübung nicht von beiden Gruppenmitglieder gleichzeitig durchgeführt, sondern zuerst von Angermann, Leo und anschließend erneut von Gössl, Sebastian. Beide führten alle Aufgaben vollständig durch und werteten ihre jeweiligen Daten, über gleiche Rechenwege, getrennt aus. Wird nur ein Ergebnis angegeben, ist es nur dies des Schriftführers (Zwischenergebnisse \& Diagramme, zu besseren Übersicht), ansonsten werden die jeweiligen Ergebnise der Laboranten nebeneinander angeführt, so dass diese verglichen werden können.



\section{Aufgabenstellung}

\td{Dieses Vorlage basiert auf dem folgenden Github-Repository: \\ \url{https://github.com/goessl/labor} \\ Ein Beispiel ist unter diesem Link zu finden.}



\section{Voraussetzungen \& Grundlagen}



\section{Beschreibung der Versuchsanordnung}


\begin{landscape}
\section{Geräteliste}

\begin{table}[H]
    \begin{tabularx}{\linewidth}{| c | c | c | c | c | c | X |}
        \hline
        \textbf{Apk.} & \textbf{Gerät} & \textbf{Hersteller} & \textbf{Typ}   & \textbf{Inventar Nr.} & \textbf{Seriennummer}  & \textbf{Anmerkung} \\
        \hline
        VM1 & Voltmeter & Mega \& Volt & MV901C/3 & - & 9705437 & Unsicherheit: 5\% \\
        \hline
        VM1 & Voltmeter & Mega \& Volt & MV901C/3 & - & 9705437 & Unsicherheit: 5\% \\
        \hline
    \end{tabularx}
    \caption{Im Versuch verwendete Geräte und Utensilien.}
    \label{tab:geraete}
\end{table}

\newpage
\begin{table}[H]
	\begin{tabularx}{\linewidth}{| c | c | c | c | c | c | X |}
		\hline
		\textbf{Apk.} & \textbf{Gerät} & \textbf{Hersteller} & \textbf{Typ}   & \textbf{Inventar Nr.} & \textbf{Seriennummer}  & \textbf{Anmerkung} \\
		\hline
		VM1 & Voltmeter & Mega \& Volt & MV901C/3 & - & 9705437 & Unsicherheit: 5\% \\
		\hline
		VM1 & Voltmeter & Mega \& Volt & MV901C/3 & - & 9705437 & Unsicherheit: 5\% \\
		\hline
	\end{tabularx}
	\caption*{\textbf{Tab. \ref{tab:geraete}(Fortsetzung)} Im Versuch verwendete Geräte und Utensilien.}
\end{table}



\end{landscape}



\section{Versuchsdurchführung \& Messergebnisse}



\section{Auswertung}

In der Auswertung werden zur erhöhten Genauigkeit durchgehend ungerundete Werte bis zu den Endergebnissen verwendet und nur zur Darstellung gerundet. \\
Zur Berechnung der Unsicherheiten wird, wenn nicht anders angegeben, die Größtunsicherheitsmethode verwendet.
\cite{dirac}asdasd;


\section{Diskussion}



\section{Zusammenfassung}



\printbibliography[heading=bibintoc]



\end{document}
